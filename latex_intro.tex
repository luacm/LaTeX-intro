\documentclass{article} % Specifies what type of document this is
% There are a couple of choices but for most small cases; article works

% usepackage is just like an import statement. We include the libraries here that we are going to work with
\usepackage[utf8]{inputenc}
\usepackage{amsmath}
\usepackage{amsthm}

\title{Intro to Latex}
\author{Michael Toth}
\date{September 2013}

% Everything above won't actually be rendered into the document. It's a place for describing how the document should be formatted
% After we "begin" the document we can actually put words on paper
\begin{document}

% Creates the title we specified above
\maketitle

% Denotes a new segemnet of the paper
\section{Introduction}

This is an introduction to some LaTeX stuff

\section{Examples}

We have the series $A_{1},A_{2},...$,
the regional sum $A_{1} + A_{2} + ... $,
the orthogonal product $A_{1},A_{2},...$,
and the infinite integral

\begin{proof}
Some very important text about life.
$\forall x \in X$
$\alpha, \beta, \gamma$

\end{proof}

\section{Cool Matrix Things}

\begin{equation}
\left(\begin{array}{cc}
1 & 2 \\0 & 1
\end{array}\right)
\left(\begin{array}{cc}
2 & 0\\1 & 3
\end{array}\right)
=
\left(\begin{array}{cc}
4 & 6 \\1 & 3
\end{array}\right)
\end{equation}

\begin{equation}
$x + y = z$
\end{equation}

\section{Other Stuff}

Subscripts $A_{5}$ \\
Superscripts $A^{5}$ \\
\(\int\!\!\int z\, dx dy .. \int\int z dx dy\)

\section{Wat}

\begin{displaymath}
\mbox{W}+\
\begin{array}{l}
\nearrow\raise5pt\hbox{$\mu+ + \nu_{\mu}$}\\
\rightarrow \pi+ +\pi0 \\[5pt]
\rightarrow \kappa+ +\pi0 \\
\searrow\lower5pt\hbox{$\mathrm{e}+
+\nu_{\scriptstyle\mathrm{e}}$}
\end{array}
\end{displaymath}

\begin{proof}
\begin{displaymath}
\frac{\pm
\left|\begin{array}{ccc}
x_1-x_2 & y_1-y_2 & z_1-z_2 \\
l_1 & m_1 & n_1 \\
l_2 & m_2 & n_2
\end{array}\right|}{
\sqrt{\left|\begin{array}{cc}l_1&m_1\\
l_2&m_2\end{array}\right|2
+ \left|\begin{array}{cc}m_1&n_1\\
n_1&l_1\end{array}\right|2
+ \left|\begin{array}{cc}m_2&n_2\\
n_2&l_2\end{array}\right|2}}
\end{displaymath}
\end{proof}

% Ends the portion of the document where we can put words on paper
\end{document}

}
